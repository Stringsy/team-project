% This example An LaTeX document showing how to use the l3proj class to
% write your report. Use pdflatex and bibtex to process the file, creating 
% a PDF file as output (there is no need to use dvips when using pdflatex).

% Modified 

\documentclass{l3proj}
\begin{document}
\title{Team V - How Not To Kill Your Dog}
\author{Ross Adam \\
        Andrew Gardner \\
        Nicole Kearns \\
        Mamas Nicolaou \\
        Asset Sarsengaliyev}
\date{18 March 2013}
\maketitle

\tableofcontents

\chapter{Conclusion}
\label{conclusion}

\section{Summary}

The aim of our project was to create a learning application for veterinary students that will allow them to learn and revise course content for learning about drug calculations in animals. After a full evaluation of our application with our client, Dr Fiona Dowell, and a group of students, we are satisfied that we met all product requirements and will integrate successfully within the veterinary school.

A student user is able to successfully and easily view all the topics available, browse through all of the slides within each topic, answering questions as they go along, and they can complete a final assessment at the end to test how much they have learned.

An admin user can successfully manage all content for the application - including uploading new slides and questions, editing topics and questions and delete any topics, slides or questions.

\section{Future Work}

% possibly 'Submit All' - but will remove the gamification.
% feedback for questions in topics.
% make user groups to avoid individually setting permissions
% use mp3s, videos
% make available on tablets, phones
% categorise final questions to allow for easy and difficult questions.
% help functionality.

After evaluating the application with our client, we have concluded that the following improvements could be made:

\begin{itemize}
\item Within the Final Assessment page, add a 'Submit All' button to avoid having to submit answers individually. However, as a consequence, this will remove the gamification element of the final assessment page so will need to be discussed further with our client.
\item Provide visual feedback when answering questions within topics, to indicate to the user if they answered correctly or incorrectly. For example, change the 'Submit' button colour to green if correct, and red if incorrect. This will be more intuitive than simply displaying 'Correct' or 'Wrong' as plain text.
\item Within the administration page, set up several custom user groups with different access permissions. This will allow the administration user to assign a new user to a group, therefore, streamlining the permissions process and eliminating current confusion.
\item Categorise all the questions within the Final Assessment page, into easy and difficult questions, having easier questions appearing at the top of the assessment. As a result, the final assessment pages will be more structured and reflect exams more accurately.
\item Add support for the inclusion of both audio and video content to create a more dynamic and interactive experience.
\item Make our application available for smart phones and tablets. This will involve making dynamic resizing pages and formatting for smaller screens.
\item Add a Help section to provide users with 'How To' instructions for each function of the application.
\item Generalise the application so that it can handle content from other courses. 
\end{itemize}

\section{Contributions}

TO BE COMPLETED!

\end{document}
