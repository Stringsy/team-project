% This example An LaTeX document showing how to use the l3proj class to
% write your report. Use pdflatex and bibtex to process the file, creating 
% a PDF file as output (there is no need to use dvips when using pdflatex).

% Modified 

\documentclass{l3proj}
\begin{document}
\title{Team V - How Not To Kill Your Dog}
\author{Ross Adam \\
        Andrew Gardner \\
        Nicole Kearns \\
        Mamas Nicolaou \\
        Asset Sarsengaliyev}
\date{18 March 2013}
\maketitle

\chapter{Design}
\label{design}

This chapter covers the various aspects of the application's design process. As our project was client-based the design was very much focused on satisfying the requirements laid out in the previous chapter. Fiona Dowell, our client from Glasgow University's Veterinary School, evaluated our initial user interface design and we then revised it according to the feedback that we received from her. When we were satisfied with the design we moved on to the implementation phase.

\newpage

\section{Design Goals and Considerations}

The overall design goals for the application follow principles already well established in the field of web development.

These were:

\begin{enumerate} 
\item{\textbf{Precedence}} \\
The use of positioning, colour, size and contrast to naturally guide the user around the application. For example, most sites have a logo featured in the top left corner because studies have shown this to be the first place users look when accessing web sites. This enables natural navigation sequences: elements like the navigation bar being strategically placed so users' eyes will follow from the larger, relatively contrasted logo to the navigation bar.

\item{\textbf{Navigation}} \\
The principle of making it easy and intuitive for users to navigate your web site. For example, links and buttons should be well positioned and clearly indicate their function while offering appropriate feedback when clicked. Measures should also be taken to inform the user where they are currently within the application.

\item{\textbf{Typography}} \\
The text of a web application is the most common element of design and therefore deserves significant consideration. Font type, size, spacing and colour are essential to the application's overall clarity and readability.

\item{\textbf{Usability}} \\
The encompassment of previously defined principles such as precedence, navigability and text. They relate to how usable the application is for its users. An important usability consideration is that of 'standards adherence'. This is the following of conventions laid out and used by millions of other websites, such as underlined text indicating a link.

\item{\textbf{Consistency}} \\
The principle of making the application's design elements ``match'' coherently between pages and on the same page of the site. Attributes like colour and font choice, for example, should remain consistent across the site.

\item{\textbf{Clarity}} \\
The appropriate use of design techniques such as positioning, contrast, font aliasing and others to make your design stand out crisply and clear. 
\end{enumerate}
	
\subsection{Admin}

The admin user is solely responsible for the content management of the application. With that in mind we approached the admin interface's design with a heavy focus on usability and navigation, rather than fancy themes and textual styles. We felt that the admin interface should possess a natural gravitas; that it should be taken seriously as a tool that controls the entire functionality of the application and not as a novel feature. To this extent our goal was to use a very simple theme, contrasting sharply in style and presentation with the student user interface. This was a wilful breaking of the consistency design principle to enforce a measure of aforementioned seriousness onto the admin interface.

We also had to consider that the likely admins of the application would be relatively inexperienced with regards to administrating an entire web application's content. Being mindful of this we had to design the interface to be accommodating for the inexperienced, allowing for easy navigation and informative feedback with every action taken.

\subsection{Student}

In stark contrast to that of the admin user interface, the student's view of the web application had to embody all design goals previously defined in this chapter. They had to be able to navigate the site easily, knowing where they were at all times and how to get to where they would like to go with no fuss or ambiguity.

As this application's core functionality is assessment based it was important to have a gentle colour theme that would contrast naturally with the displayed content and not distract the user. This also held true with appropriate user feedback. We made an early design decision to provide the user with timely responsive feedback when answering questions.

An important design goal was to create an interactive learning experience for the user. In order to achieve this we decided to include an animal avatar with an associated health bar. A correct answer would have no effect whereas an incorrect answer would result in health loss. This design feature helps to engage the student on a personal level as their progress will affect the animal's lifespan. 

\section{Initial Design}

\subsection{ER Diagram}
	
After establishing our initial design goals we created an entity relationship diagram. Defining our data models at this phase allowed us to clearly understand the distinction between entities and how they relate to each other.



\subsection{Activity Flow Diagrams}
	
This diagram shows the flow of interaction for each user type within the application. 

\subsection{Walkthrough}
	
Walkthrough linking the activities of admin and user together.

\subsection{Paper Prototypes}

Paper prototyping was an effective and inexpensive method used to determine the design direction of the student user interface. The use of paper prototypes allowed us to experiment with various design configurations without the overhead of time and resources. This method of design was particularly suited to this application as it was simple to review and to iterate upon. Client involvement was a requirement for the success of the project. Therefore it was imperative that our early design ideas were subject to client review and feedback. Sending several prototypes for client evaluation meant that there was more potential for varied, constructive feedback. Receiving said feedback from the client afforded us more opportunity to revise our designs in accordance with their wishes. 

The paper prototypes aimed to fulfil the student user's functional requirements as laid out in the previous chapter. This meant that we had to design mechanisms for the following use cases:

\begin{itemize}
\item{Viewing Topics\textbf{ (SR1)}}
\item{Viewing Slides\textbf{ (SR2)}}
\item{Viewing Questions\textbf{ (SR3)}}
\item{Answering Questions\textbf{ (SR4)}}
\end{itemize}

As will be discussed in the Prototype Evaluation section, our client indicated that two prototypes were particularly suitable in meeting the requirements of the application.

\textbf{Figure 1.1} --> Topic/Slide View 		\textbf{Figure 1.2} --> Final Assessment Page

The above-left image (\textbf{Fig 1.1}) shows the individual topic view for the student user interface. The ``title'' at the top of the page represents the topic's name, with the slides for the topic being shown in the ``content'' section. The arrows contained within the content section represents the possibility of a slider being used, enabling the user to easily switch back and forth between slides. The breadcrumbs, shown below the content box, were designed to enhance user navigability by allowing the user to directly select a slide to navigate to. The rounded buttons to the right of the content section represent clickable links to navigate between individual topics. These design elements satisfy requirements \textbf{SR1} and \textbf{SR2}.

This prototype satisfies design goals laid out in the beginning of this chapter. The judicious use of whitespace enhances the user interface's clarity. With regards to design precedence, the size of the content box (compared to other design features) functions as a focal point, demanding the user's attention. 

The above-right image (\textbf{Fig 1.2}) shows the final assessment page for the student user interface. The inclusion of an animal avatar and health bar satisfies the design goal of interactive student engagement, resulting in a heightened learning experience. Similar to \textbf{Figure 1.1}, the use of whitespace and size precedence allows the user to understand and easily navigate the interface presented to them. The problem description section represents the functional requirement \textbf{SR3} and the box below plus the apply/inject button represents the functional requirement \textbf{SR4}.  

We chose not to create a paper prototype of the admin interface because we knew that Django (the web framework we had chosen to implement the application with) would provide a suitable pre-defined interface. This will be discussed further in the Admin User Interface Subsection. 
 
\subsection{Prototype Evaluation}

The evaluation process involved sending the paper prototypes to our client, Fiona Dowell. This was an important part of the design process as her feedback would naturally impact the application's final design. 

With regards to the paper prototypes, the feedback confirmed our initial design choices. Both designs preferred by our team were also chosen by Fiona and her colleagues. The decision to revise and continue with these designs was reinforced by our client's feedback. This not only resulted in an easy design decision but also supported the overall design direction we had previously agreed on. While we knew further iteration and changes were likely required for implementation, we felt this was a solid base to start from.

\section{Interface Design}

This section covers the user interfaces for both admins and students, analysing how the design meets specific functional requirements and overall design goals.

\subsection{Admin User Interface}

<annotated front page diagram>

Figure (x.x) shows the main page of the admin interface after the admin user has logged in. The interface's specific design theme was largely out of our hands as Django provided this for us. The lack of design control ofer the interface wasn't an issue however, as it conformed to our earlier design goals that the interface should be sufficiently different in theme and style from the student user interface. As figure (x.x) shows, the overall theme is quite minimalistic. There is no extraneous details or features that could distract the admin use

How does the admin user interface design match its design and functionality goals?
Using annotated screenshots, show the different parts of the admin interface.
Talk about how little things like the action history allows an inexperienced admin user to understand and review past actions.

Issues: We didn't have fine grained control over the look and feel of the admin interface. While we still felt it was appropriate and suitably met our design considerations, it would have been better if Django allowed us greater customisation.

\subsection{Student User Interface}

1. Show and talk about the welcome screen. (annotated)
Compare prototype to welcome screen. How does it compare? What are the improvements? What is it about the welcome screen that makes it suitable in the overall design considerations / functional requirements.

if Issues: What they are, why they are and how they were overcame or could be prevented.

2. Show and talk about the topic/question screen. (annotated)
Compare prototype to welcome screen. How does it compare? What are the improvements? What is it about the welcome screen that makes it suitable in the overall design considerations / functional requirements.

if Issues: What they are, why they are and how they were overcame or could be prevented.

2. Show and talk about the final assessment screen. (annotated)
Compare prototype to welcome screen. How does it compare? What are the improvements? What is it about the welcome screen that makes it suitable in the overall design considerations / functional requirements. Relate to this being part of a fun learning environment. How the animal life bar links you tentatively to care about the creature. That kinda thing.

if Issues: What they are, why they are and how they were overcame or could be prevented.
One issue was feedback from Fiona concerning the size of slider and question placement. Say how the final design reflects this and how useful it was to have a client that was able to provide assertive feedback on what she required. 

\section{Conclusion}

Did the overall design meet the individual requirements, functional requirements and overall design considerations?
Where they didn't, how could this be improved? i.e. what would we change about the design process to nullify these issues.
Were we pleased with the overall outcome of the design?
One of the factors of design pleasure was the client.
How does the achivement of a sucessful design process prepare you for a successful implementation?
Did we feel this was the case?




\end{document}
