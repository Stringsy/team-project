% This example An LaTeX document showing how to use the l3proj class to
% write your report. Use pdflatex and bibtex to process the file, creating 
% a PDF file as output (there is no need to use dvips when using pdflatex).

% Modified 

\documentclass{l3proj}
\begin{document}
\title{Team V - How Not To Kill Your Dog}
\author{Ross Adam \\
        Andrew Gardner \\
        Nicole Kearns \\
        Mamas Nicolaou \\
        Asset Sarsengaliyev}
\date{18 March 2013}
\maketitle
\begin{abstract}

This project aims to produce a learning application to be used within the University of Glasgow School of Veterinary Medicine by students wishing to revise and test their knowledge.  The Django Web Framework was used to produce a web based application that is hosted in the veterinary school. It consists of both teaching slides with course content and affiliated questions. There is also administration functionality to allow the course co-ordinator to edit the content of the application to evolve with the prescribed course content.

\end{abstract}
\educationalconsent
\tableofcontents

\chapter{Introduction}
\label{intro}



\section{Motivation}
Team V is comprised of five students currently in their third year of a Computing Science degree at the University of Glasgow. The team will be working for 7 months on this project which is the sole piece of coursework for the Team Project 3 course. By the end of this period of time it is hoped that there will be a fully functional piece of software that can be delivered to our client.
\newline
\newline
This project is ``How not to kill your dog" a macabre title for our veterinary student revision program. Our client is Dr Fiona Dowell, a senior lecturer at the University of Glasgow's Veterinary School. In addition to our client there are several more stakeholders for this application: the veterinary students who will be using this application throughout their studies; other lecturers and finally I.T. support staff who will interact with this application extensively.  Dr Dowell found that her students struggle with learning how to do drug calculations more than they do with any other part of their course. She came up with the idea of having an application available to veterinary students to help enhance their drug calculation skills.
\newline
\newline
Dr Dowell believes that students will find it more entertaining to have a game-like learning application, which they could use in their free time to entertain themselves, but also learn and become better with their drug calculations. She would therefore like a program that can be used as a revision aid for students which she can add her own educational slides to and create tests and questions to engage students.
\newline
\newline
The motivation for developing this software was to bring all the education resources currently in use and create a more centralised location. Currently the client simply uses PowerPoint slides filled with information which her students can download from their respective Moodle site. Using our software would allow the user to forgo the need for specific software to read .ppt extension files and reduce compatibility issues. Also users do a lot of calculations and all tests with pen and paper; with new software it would be hoped that calculations could be performed online which would automatically compare them and tests would be submitted online for marking. This would result in faster and more efficient work for both students and lecturers. Dr Dowell also considered the possibility of promoting this application outside of the the veterinary school in hopes of further enhancing the reputation of the University of Glasgow.

\section{Project Aim}

The aim of the project was to create a learning application for veterinary students that allows them to learn new content and revise and refresh existing knowledge. As this project has been heavily client dependant the team has had several meetings with Dr Dowell to ensure that high quality software that meets her requirements has been produced.
\newline

Through this process we have identified these aims for the project:
\newline

\begin{itemize}
\item To centralise existing revision material.
\item Provide a system students can use to learn and revise course content.
\item Provide a system which can include informal test material to assure students of their knowledge.
\item Allow the evolution of the application content to match course content through the adding, deleting and editing of topic slides and questions.
\item Make the learning process engaging and enjoyable for users.
\end{itemize}   

\section{Dissertation Outline}

The dissertation outlines in detail the process of carrying out the ``How Not To Kill Your Dog" project. Below is an outline of our report:

Chapter 2 discusses existing applications and similar revision tools which we found useful in the design process.

Chapter (Requirements) discusses the requirements gathering process and outlines the functional and non-functional requirements for our application.

Chapter (Design) discusses the general design of the application.

Chapter (Implementation) discusses the implementation of the project.

Chapter (Testing)

Chapter (Evaluation)

\end{document}
