% This example An LaTeX document showing how to use the l3proj class to
% write your report. Use pdflatex and bibtex to process the file, creating 
% a PDF file as output (there is no need to use dvips when using pdflatex).

% Modified 

\documentclass{l3proj}
\begin{document}
\title{Team V - How Not To Kill Your Dog}
\author{Ross Adam \\
        Andrew Gardner \\
        Nicole Kearns \\
        Mamas Nicolaou \\
        Asset Sarsengaliyev}
\date{18 March 2013}
\maketitle
\begin{abstract}

The abstract goes here

\end{abstract}
\educationalconsent
\tableofcontents

\chapter{Implementation}
\label{implementation}
\section{Abstract}
\section{Developing a Web Based Application }
One of the requirements for our project was that our final product should be available to 
use to any user with an Internet Connection. More specifically, any student should be 
able to access the application and practice on their drug calculations in their free time 
wherever they are with any device that can access the Web. This can be a laptop 
computer, a mobile phone or a tablet computer. \\ 
It was, therefore, clear to us that creating an application that needs to be installed on a 
machine in order to be usable was not an option. After a discussion with the team we 
decided that the most suitable solution would be to implement a application that would be 
accessible via a Web Browser. A web application would make it possible for any user to 
access our application from any device simply by navigating to our Web Applications 
URL address. \\ 

\subsection {Using a Web Application Framework}
After doing some research looking for the best possible option for developing web 
applications we decided that we would make use of a web development framework 
rather than writing the code for the whole application from scratch.
\subsubsection{ Developing from Scratch}
Developing an application from scratch has the potential to take up much needed time 
from the Development Phase. For each different page of an application there needs to be 
a unique file (for example an html document) that will be sent to the client's web browser 
when the page is called. An application usually has the same layout on every page with 
basic components, like the applications logo or the applications default buttons, being 
displayed on each page. It is therefore clear that having a separate file to correspond to 
each different page of an application can lead to a lot of coding overhead and 'boiler 
plate' code. In the case of a change request, a developer will need to modify the code in 
each different page separately which would take much of our development time. Time was a major concern for us as we only had a limited amount of time to get the application designed, developed, evaluated and delivered to our client.
\subsubsection{Developing with a web application framework}
sdfdsth
\subsection {Which framework to use?}
Comparison between Ruby on Rails, Web2Py and Django
\subsection {Why Django}
Reasons we have chosen Django and the advantages it offers.
\section{Development in Django}
Brief description of django
\subsection{MVT Architecture}
Describe the MVT Architecture.
\subsection{Models}
Describe the role of Models in Django
\subsection{Views}
Describe the role of Views in Django
\subsection{Templates}
Describe the role of Templates in Django
\subsection{Controller}
Talk about the URL Dispatcher/Django Framework
\section{3-Tier Architecture}
Describe the 3-Tier Architecture used for our project
\subsection{N-Tier Architecture Diagram}
Architecture Diagram image goes here
\subsection{Front End}
Explain the front end part of the 3-tier architecture
\subsection{Middleware}
Explain the middleware part of the 3-tier architecture
\subsection{Back End}
Explain the back end part of the 3-tier architecture
\section{Back End}
Go into more detail about the Back End. Talk about the database, the models used, their default values, the database schema etc.
\section{Managing the Front End}
Describe how the front end is managed through django. 
\subsection{Possible frameworks for Front End management}
\subsubsection{Backbone}
Brief description of backbone.js
\subsubsection{Tastypie}
Brief description of tastypie.js
\subsubsection{Pyjamas}
Brief description of Pyjamas
\subsubsection{Compination of traditional Django friendly frameworks}
Describe how traditional technologies such as javascript, jquery and Ajax work with Django and why these are selected over other technologies mentioned before.
\section{Middleware: Linking the back with the front}
Talk about how the Django server makes the connection between the front end and the back end.
\section{Message Passing}
\subsection{Database Request Format}
Database requests description and sequence diagram
\subsection{Answer Validation Request}
Answer validation request description and sequence diagram
\subsection{Topic Related Data Request}
Topic Related Data request and sequence diagram
\section{End Product}
\subsection{Desired Functionality}
How the desired functionality mentioned in Requirements and Design sections was realised through the implementation
\subsubsection{Welcome Page}
How the Welcome page is implemented
\subsubsection{Topic Page}
How the topic page is implemented
\subsubsection{Contents Page}
How the content page is implemented
\subsubsection{Final Assessment Page}
How the final assessment page is implemented
\subsubsection{Administration Page}
How the administration page is implemented
\subsection{Interaction Diagrams}
\subsubsection{Topic Page}
Interaction diagram for communications and message passing between the Topic page and the server.
\subsubsection{Contents Page}
Interaction diagram for communications and message passing between the Contents Page and the server
\subsubsection{Final Assessment Page}
Interaction diagram for communications and message passing between the Final Assessment Page and the server
\section{Challenges and Solutions}
Talk about the risks and challenges faced during the development phase of the project and how these were faced.
\section{Known Issues}



\end{document}
