 % This example An LaTeX document showing how to use the l3proj class to
% write your report. Use pdflatex and bibtex to process the file, creating 
% a PDF file as output (there is no need to use dvips when using pdflatex).

% Modified 

\documentclass{l3proj}
\begin{document}
\title{Team V - How Not To Kill Your Dog}
\author{Ross Adam \\
        Andrew Gardner \\
        Nicole Kearns \\
        Mamas Nicolaou \\
        Asset Sarsengaliyev}
\date{18 March 2013}
\maketitle
\tableofcontents

\chapter{Evaluation and Testing}
\label{evaluation}

\section{Testing}

\section{Evaluation}

In order to test the usability and functionality of our application, we created an evaluation plan with a set of tasks for the user to carry out. The evaluation plan included tasks for both the admin user and the student user as this allowed us to evaluate all areas of our application.

Before carrying out our evaluation, our team carried out the evaluation on ourselves in order to test the application for issues and see how many of the requirements we satisfied. From this evaluation, we found that of the specified functional requirements, our application satisfies all of them. The admin is able to upload new content, edit content and delete content. The student user can view all topics available and see all related slides and questions for each topic. The stuent user can also complete the final assessment and see their final score, indicating how well they done.

\textbf{Users}

Firstly, we wanted to an evaluation with our client, Dr Fiona Dowell. This would allow us to see how satisfied our client was with the application we had created and allow us to receive any feedback. Following that, we wanted to get a mix of Computing Science/Software Engineering students and other students to evaluate our application. As the computing science/Software engineering students had done coursework with Django recently, we thought that having participants who are unfamiliar with Django would provide us with the best results from our evaluation.

\textbf{Features To Test} 

we wanted to evaluate as many features as possible:

\begin{itemize}
\item Content management; admin users should be able to easily upload, edit and delete content.
\item Browse through topics successfully
\item Answer all questions available
\item Complete the final assessment and see how well they done.
\end{itemize}

\textbf{Tasks}

We asked the participants to carry out a number of tasks in order to evaluate our application. We have both tasks for an admin user and for a student user which allows us to evaluate all aspects of our application.

\textbf{Admin User Tasks}


\begin{tabular}{|c|c|}
\hline & \textbf{Task}\\
\hline
\hline 1 & Login\\
\hline 2 & Add a new user\\
\hline 3 & Set the users permissions to allow them to update content.\\
\hline 4 & Add a new Topic.\\
\hline 5 & Add a new slide to that topic.\\
\hline 6 & Go to Topic 2 and remove slide 3.\\
\hline 7 & Add a new question to Topic 1.\\
\hline 8 & Edit question 2 within Topic 2.\\
\hline
\end{tabular}

%\begin{enumerate}
%\item Login
%\item Add a new user.
%\item Set the new users permissions to allow them to update content.
%\item Add a new topic.
%\item Add a new slide to that topic
%\item Go to Topic 2 and remove slide 3
%\item Add a new question to Topic 1
%\item Edit question 2 within Topic 2
%\end{enumerate}

\textbf{Student Tasks}

\begin{tabular}{|c|c|}
\hline & \textbf{Task}\\
\hline
\hline 1 &  Go to Topic 1 and browse through the slides.\\
\hline 2 & Answer 2 questions within Topic 1.\\
\hline 3 & Go to Topic 2.\\
\hline 4 & Go to the final Assessment and complete the test.\\
\hline
\end{tabular}

%\begin{enumerate}
%\item Go to Topic 1 and browse through the slides.
%\item Answer a few questions within Topic 1
%\item Go to Topic 5
%\item Go to Topic 2
%\item Go to final assessment and complete the test.
%\end{enumerate}

\textbf{What will be measured}

\begin{itemize}
\item Usability
\item Perfomance
\item Success Rate
\end{itemize}

\textbf{Think-aloud}
%User observed performing tasks and asked to desccribe what he is doing and why, what he thinks is happening.
%Advantages - simplicity; can provide useful insight; can show how the system is actually used
%Disadvantage - subjective; selective; act of describing may alter task performance

All evaluations were performed using the Think Aloud method which involves observing the participants performing the tasks and having them describe out loud what they are doing and why. This is a useful evaluation method as it


\textbf{Questionnaire and Additional Feedback}

After the evaluation, each participant was asked to complete a quick questionnaire about the system and provide any additional feedback. 

\subsection{Ethics}

\subsection{Results}


\end{document}
